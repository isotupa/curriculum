\documentclass[letterpaper]{cv} % a4paper for A4

%----------------------------------------------------------------------------------------
%	 LEFT RECTANGLE
%----------------------------------------------------------------------------------------

\cvname{MARKO ISOTUPA CAÑIZAL} % Your name
\cvjobtitle{ Ingeniero Informático} % Job

% datos de contacto
\contact{\\
\textbf{Teléfono}\\
(+34) 653 56 65 55

\textbf{Localidad}\\
Madrid, España

\textbf{Correo electrónico}\\
marko.isotupa01@gmail.com

\textbf{GitHub}\\
{\href{github.com/isotupa}{github.com/isotupa}}

\textbf{LinkedIn}\\
linkedin.com/in/marko-isotupa
}

% Idiomas
\idiomas{{\textbf{Francés} B2 (DELF) / 3}, {\textbf{Finlandés} Nativo / 6}, {\textbf{Inglés} C2 (Cambridge) / 6}, {\textbf{Español} Nativo / 6}}

% Educación
\education{
\textbf{Grado en Ingeniería Informática}\\
Universidad Politécnica de Madrid \\
2019 - 2023 | Madrid, España

\textbf{Bachillerato Tecnológico}\\
Kensington School \\
2017 - 2019 | Madrid, España
}

% acción social
\accionsocial{
\textbf{Reparación de colegio y clases de inglés en Tailandia}
{\begin{itemize}
        \item Adquisición de competencias de trabajo en equipo para coordinar la repara-ción de un colegio.
        \item Enseñar inglés a una clase de cerca de 50 niños de bajo nivel.
\end{itemize}}
}

%----------------------------------------------------------------------------------------

\begin{document}

\makeprofile % Print the sidebar
 
%----------------------------------------------------------------------------------------
%	 EXPERIENCE
%----------------------------------------------------------------------------------------

\color{black!70}
\section{Experiencia}

\begin{twenty} % Environment for a list with descriptions
\twentyitem
    	{Sep 2023 -}
		{Presente}
        {Investigador}
        {CVAR (Computer Vision and Aerial Robotics}
        {}
        {\begin{itemize}
        \item Cooperación de proyectos de investigación relacionados con visión computacional aplicado drones con el fin de automatizarlos.
        \item Implementación de tecnologías de reconocimiento facial, de gestos y de posturas para modificar el comportamiento de drones autónomos.
        \item Trabajo con simuladores y herramientas de control de robots.
        \end{itemize}}
        \\
	\twentyitem
    	{Mayo 2020 -}
		{Presente}
        {Responsable del observatorio}
        {Obsevatorio Astronómico UPM}
        {}
        {
        {\begin{itemize}
        \item Puesta en marcha del observatorio astronómico, arreglándolo y manteniéndolo por iniciativa propia.
        \item Liderar tres equipos de unos diez estudiantes para planear y ejecutar eventos en el observatorio.
        \item Organización de cerca de diez eventos de observación de estrellas.
    \end{itemize}}
        }
    \\   
    \twentyitem
   		{Sep 2021 -}
		{Jun 2023}
        {Azafato de eventos}
        {Múltiples empresas}
        {}
        {
        {\begin{itemize}
        \item Coordinación y asistencia en la ejecución de más de veinte eventos distintos de manera eficiente.
        \item Trabajo en equipo y colaboración en situaciones de alta presión.
    \end{itemize}}
        }
        
	%\twentyitem{<dates>}{<title>}{<location>}{<description>}
\end{twenty}

%----------------------------------------------------------------------------------------
%	 HABILIDADES DE TIC
%----------------------------------------------------------------------------------------
\section{Habilidades de TIC}
\textbf{Lenguajes de programación}\\
\vspace{5pt}
Java $\textbullet$ C $\textbullet$ Python $\textbullet$ R $\textbullet$ Assembly $\textbullet$ MatLab $\textbullet$ Bash $\textbullet$ Prolog $\textbullet$ Lua\\
\textbf{Herramientas}\\
\vspace{5pt}
JUnit Testing $\textbullet$ Android Studio $\textbullet$ VirtualBox $\textbullet$ GNU/Linux $\textbullet$ redes IP $\textbullet$ Git $\textbullet$ Docker\\
\textbf{Administración de Bases de Datos}\\
\vspace{5pt}
MySQL $\textbullet$ JDBC\\
\textbf{Herramientas de ML/AI}\\
\vspace{5pt}
Keras $\textbullet$ Matplotlib $\textbullet$ Pandas $\textbullet$ OpenCV\\
\textbf{Metodologías}\\
Agile $\textbullet$ SixSigma\\

%----------------------------------------------------------------------------------------
%	 PROYECTOS
%----------------------------------------------------------------------------------------

\section{Proyectos}
\textbf{\underline{\smash{Implementación de una Red Neuronal en C desde Cero}}}
{\begin{itemize}
        \item Desarrollé una red neuronal desde cero utilizando el lenguaje de programación C, con el objetivo de profundizar en la comprensión del funcionamiento interno de las redes neuronales.
        \item Logré que la red neuronal fuera capaz de entrenar con conjuntos de datos específicos y realizar predicciones precisas.
        \item \textbf{Herramientas usadas:} C, Redes neuronales, ML
\end{itemize}}
\textbf{\underline{Desarrollo de Minishell de Unix en C}}
{\begin{itemize}
        \item Diseñé un intérprete de comandos capaz de ejecutar programas, gestionar señales, manejar redirecciones, pipes y ejecución en segundo plano, entre otras funcionalidades esenciales de un shell Unix.
        \item \textbf{Herramientas usadas:} C, compiladores, UNIX, Bash, Sistemas Operativos
\end{itemize}}
\textbf{\underline{\smash{Control de Drones mediante Gestos Captados por Cámara Integrada}}}
{\begin{itemize}
        \item Concebí, diseñé e implementé un sistema que permitía controlar drones utilizando gestos reconocidos por la cámara integrada en el dron, combinando visión por computador y tecnologías de vuelo autónomo.
        \item \textbf{Herramientas usadas:} Python: OpenCV, MediaPipe... Linux, simuladores, Docker
\end{itemize}}


\end{document} 
